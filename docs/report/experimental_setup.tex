\subsection{Origin of the data}
  The data and information is coming from PMRF, an instrumented US Navy testing range located off the island of Kauai, Hawaii. Data were collected from seven bottom mounted hydrophones (4 to 5 m off the seafloor) in deep water (nominally 4,600 meters) approximately 45 km northwest of Kauai. The relative location of the hydrophones, their designations and filenames for the data files are as shown in Table \ref{tab:hydrophones_positions}. Our targets were always minke whales.
  
  \begin{table}
    \caption{Position of hydrophones}
    \label{tab:hydrophones_positions}
  
    \begin{center}
      \begin{tabular}{| l | l | l | l | l |}
         \hline
         hydrophone \# & X[m] & Y[m] & Z[m]  & filename \\ \hline
         7 \# & -6129 & 9784 & -4750  & 27Apr09\_074921\_NN\_p7.wav \\ \hline
         6 \# & -6183 & -4874 & -4650  & 27Apr09\_074921\_NN\_p6.wav \\ \hline
         5 \# & -6163 & -12402 & -4600  & 27Apr09\_074921\_NN\_p5.wav \\ \hline
         4 \# & 6865 & 12844 & -4750  & 27Apr09\_074921\_NN\_p4.wav \\ \hline
         3 \# & 6520 & 5240 & -4650  & 27Apr09\_074921\_NN\_p3.wav \\ \hline
         2 \# & 6635 & -2132 & -4500  & 27Apr09\_074921\_NN\_p2.wav \\ \hline
         1 \# & 6566 & -9617 & -4500  & 27Apr09\_074921\_NN\_p1.wav \\ \hline
      \end{tabular}
    \end{center}
  \end{table}
  
  The dataset is from the 27th of April 2009 at approximately 12:00 Hawaiian standard time. Thirty minutes of data from each hydrophone are provided as three files (approx. 10 minutes per file). The sampling rate of the recordings (Windows PCM .wav format) is $Fs=\SI{96}{\kilo\Hz}$ with 16 bits resolution. The data are sampled simultaneously, so sample N from one file is the same relative time as sample N of a second file with the same filename. The file format is .wav in little-endian format.
  
  The test signals used to assess the proper implementation of the algorithms were two chirps, two pure tones and white noise delayed a desired number of samples. The implementation of their creation can be seen in file \emph{init\_test\_signals.m}\cite{inittestsignals.m}.
  
\subsection{Procedure}
  First, two signals are loaded and cut to focus at an interesting event. We've used Audacity to visually discover the minke whale sounds. The length of the window to cut the signals must be at least $TDOA_{expected}+L_{signal}$.
  
  One one hand, the worst case for $TDOA_{expected}$ is when the source is aligned with both microphones. Then, the maximum possible delay between to pair of microphones will be $\frac{\text{distance}}{c}$, where $c$ is the speed of sound underwater. $c$ is always considered constant, altough better approximations could have been used, taking into consideration temperature, salinity and pressure (depth) \cite{speed-sound-seawater}. On the other hand, a minke whale sound ($L_{signal}$) lasts around 6 seconds. Hence, our window length will be very large due to the huge distance between sensors. In our case, $TDOA_{max}$, corresponds to the delay between the furthest microphones. Such value has been computed using the code in the file \emph{maxdelay.m} \cite{maxdelay.m}. The huge distance between sensors (many kilometers) leads us to a window length of 20-25 s. This will a problem when we run the adaptive algorithms.

  Then, after having cut the signal, we pre-processed it with a noise removal algorithm. Finally, we proceed to the TDE between the two signals. Knowing the true delay, we only have to do a ground truth assess. The true delay is easily visualized using Audacity. Then, we record the results as relative and absolute error in samples.
  
  The comparison between TDE algorithms and preprocessing methods has been carried out using sensor 1 and 2 and event \#1 (described below). The last step, localization, which is discussed later, have been tried between all combination of sensors and event \#1.

\subsection{Some events}
  We focused on two events:
  \begin{itemize}
    \item First event: 2:20-2:45. Only noise, no other interference than the constant tone at much higher frequencies.
    \item Second event: 8:40-9:00. It has interferences from other animals (dolphins and whales). For this reason, it's more critical and more important to reduce strongly the noise and interference. The results of the simulation depend heavily on the algorithms of noise reduction and TDOA used.
  \end{itemize}
