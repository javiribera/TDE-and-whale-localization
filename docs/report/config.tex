\usepackage{cite}
\usepackage[utf8]{inputenc}
\usepackage{siunitx}
\usepackage[pdftex]{graphicx}
\usepackage{mathtools}
\usepackage{bm}
\usepackage{breqn}
\usepackage{enumerate}
\usepackage{listings}
\usepackage{textcomp}
\usepackage[colorlinks=true,urlcolor=blue,linkcolor=red]{hyperref}
\lstset{
  basicstyle=\footnotesize,
  frame = lines
}


% *** MATH PACKAGES ***
%
%\usepackage[cmex10]{amsmath}
% A popular package from the American Mathematical Society that provides
% many useful and powerful commands for dealing with mathematics. If using
% it, be sure to load this package with the cmex10 option to ensure that
% only type 1 fonts will utilized at all point sizes. Without this option,
% it is possible that some math symbols, particularly those within
% footnotes, will be rendered in bitmap form which will result in a
% document that can not be IEEE Xplore compliant!
%
% Also, note that the amsmath package sets \interdisplaylinepenalty to 10000
% thus preventing page breaks from occurring within multiline equations. Use:
%\interdisplaylinepenalty=2500
% after loading amsmath to restore such page breaks as IEEEtran.cls normally
% does. amsmath.sty is already installed on most LaTeX systems. The latest
% version and documentation can be obtained at:
% http://www.ctan.org/tex-archive/macros/latex/required/amslatex/math/


% correct bad hyphenation here
\hyphenation{op-tical net-works semi-conduc-tor matlab}

\newcommand{\vect}[1]{\mathbf{#1}}


\newcommand{\balancecolsandclearpage}{
  \close@column@grid
  \clearpage
  \twocolumngrid
}
